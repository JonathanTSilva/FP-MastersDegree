\chapter{Resultados e Discussões} \label{cap:resultados}

\section{Resultados Esperados}

Na busca por aprimorar a cibersegurança dos sistemas de controle e automação industrial por meio de uma análise meticulosa das vulnerabilidades em redes OPC UA, é imperativo delinear os resultados esperados da metodologia aplicada no presente trabalho. As expectativas de resultados estão fundamentadas em uma avaliação abrangente da robustez da rede e variações no desempenho dos controladores ao serem submetidos aos cenários de ataque cibernético apresentados na \autoref{sec:attacks}.

Primeiramente, no que se refere à simulação de ataques de \textit{Packet Sniffing}, espera-se que as redes OPC UA demonstrem um alto nível de resistência à interceptação não autorizada de pacotes, decorrente do modo de segurança inerente ao protocolo utilizado. O maior nível de proteção é apresentado pelo modo \textbf{Sign\&Encrypt}, no qual inclui recursos de criptografia e autenticação. Consequentemente, as informações trocadas entre cliente e servidor OPC UA permanecem confidenciais, íntegras e disponíveis (CIA), garantindo assim a segurança da rede.

Em segundo lugar, no contexto de ataques do tipo \textit{Man-in-the-Middle} (MITM), é imperativo considerar a detecção e prevenção destas intrusões, também com uma dependência significativa do modo de segurança selecionado. Semelhante ao cenário de ataque anterior, uma configuração que priorize o mais alto nível de segurança e a seleção adequada de políticas de criptografia devem, a princípio, proteger a rede OPC UA contra ataques MITM perpetrados por um possível Elemento Invasor. Entretanto, em situações que existam vulnerabilidades conhecidas na rede e em sua configuração, como a utilização do modo \textbf{None}, um elemento não confiável pode explorar tais fragilidades para corromper a tabela ARP (\textit{ARP Spoofing}), permitindo a interceptação das informações transmitidas entre o cliente e o servidor. Além disso, esse invasor pode modificar potencialmente dados por meio da inserção de algum \textit{malware}.

Por fim, no que diz respeito a ataques de negação de serviço (DoS), é importante considerar que os resultados esperados podem diferir dos observados nos ataques mencionados anteriormente, dependendo da capacidade da rede e de processamentos dos componentes \textit{hosts} do servidor OPC UA. Antecipa-se que, embora o ambiente experimental desenvolvido compreenda apenas alguns dispositivos de redes e não incorpore preocupações com a capacidade de comunicação, a correta avaliação dos dados capturados nos cenários simulados deverá evidenciar que esse ataque pode prejudicar a troca de mensagens ao esgotar os recursos de uma rede com grande composição. Estima-se, ainda, que os danos sejam mais significativos quando se utiliza o modo de segurança \textbf{Sign\&Encrypt} e quando o inunda com os pacotes referentes a validações de certificado e ao processo de criptografia.

Além disso, a pesquisa se esforça para fornecer informações valiosas sobre vulnerabilidades potenciais que podem ser expostas durante o processo de experimentação. Estas prospecções, caso confirmadas, auxiliam em avanços futuros do protocolo OPC UA e de sistemas IACSs, fortalecendo ainda mais a robustez destes e resistência contra ameaças cibernéticas em constante evolução.

