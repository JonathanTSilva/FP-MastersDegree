%% USPSC-Resumo.tex
\setlength{\absparsep}{18pt} % ajusta o espaçamento dos parágrafos do resumo		
\begin{resumo}
	\begin{flushleft} 
			\setlength{\absparsep}{0pt} % ajusta o espaçamento da referência	
			\SingleSpacing 
			\imprimirautorabr~~\textbf{\imprimirtituloresumo}.	\imprimirdata. \pageref{LastPage}p. 
			%Substitua p. por f. quando utilizar oneside em \documentclass
			%\pageref{LastPage}f.
			\imprimirtipotrabalho~-~\imprimirinstituicao, \imprimirlocal, \imprimirdata. 
 	\end{flushleft}
\OnehalfSpacing

O crescimento avançado da transformação digital, uma consequência direta da quarta revolução industrial, tem impulsionado de forma substancial a interconexão entre dispositivos e sistemas, introduzindo desafios significativos no âmbito da proteção de dados e sistemas críticos. A mudança de paradigma que caracteriza a convergência entre as tecnologias de informação e operacional requer uma análise criteriosa, sobretudo na indústria, cujos sistemas de automação e controle desempenham um papel central. Neste contexto, o protocolo OPC UA emerge como uma peça-chave ao permitir a transferência segura de informações entre uma variedade de dispositivos e sistemas. Não obstante, considerando a constante evolução das ameaças cibernéticas, é imprescindível adotar uma abordagem proativa na identificação e mitigação de vulnerabilidades. Esse estudo apresenta uma análise de vulnerabilidades em redes OPC UA, por meio da implementação de uma bancada experimental especialmente concebida para simular ataques cibernéticos. Os resultados da análise realizada na execução dos ataques de \textit{sniffing} de pacotes, MITM e DoS, reforçam a resiliência destas redes e, simultaneamente, contribuem para o contínuo aprimoramento do protocolo em questão. Assim, a pesquisa desempenha um papel crucial ao proporcionar percepções valiosas e abordagens concretas para a proteção de IACSs em um ambiente caracterizado por uma rápida evolução tecnológica.
 
 \textbf{Palavras-chave}: Segurança Cibernética. OPC UA. Vulnerabilidades. Ataques. IACS.
\end{resumo}