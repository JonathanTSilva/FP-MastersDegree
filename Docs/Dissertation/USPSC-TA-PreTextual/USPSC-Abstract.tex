%% USPSC-Abstract.tex
%\autor{Silva, M. J.}
\begin{resumo}[Abstract]
 \begin{otherlanguage*}{english}
	\begin{flushleft} 
		\setlength{\absparsep}{0pt} % ajusta o espaçamento dos parágrafos do resumo		
 		\SingleSpacing  		\imprimirautorabr~~\textbf{\imprimirtitleabstract}.	\imprimirdata.  \pageref{LastPage}p. 
		%Substitua p. por f. quando utilizar oneside em \documentclass
		%\pageref{LastPage}f.
		\imprimirtipotrabalhoabs~-~\imprimirinstituicao, \imprimirlocal, 	\imprimirdata. 
 	\end{flushleft}
	\OnehalfSpacing 
   The rapid growth of digital transformation, a direct consequence of the Fourth Industrial Revolution, has substantially driven the interconnection of devices and systems, introducing significant challenges in the realm of data protection and critical system security. The paradigm shift characterizing the convergence of information and operational technologies demands careful scrutiny, particularly within the industrial sector, where automation and control systems play a central role. In this context, the OPC UA protocol emerges as a pivotal component, enabling secure information transfer among a variety of devices and systems. Nevertheless, given the constant evolution of cyber threats, it is imperative to adopt a proactive approach to identify and mitigate vulnerabilities. This study presents a meticulous analysis of vulnerabilities in networks OPC UA, implemented through a specially designed experimental setup to simulate cyberattacks. The results of the analysis, conducted through packet sniffing, Man in the Middle (MITM), and Denial of Service (DoS) attacks, reaffirm the resilience of these networks while simultaneously contributing to the ongoing improvement of the protocol in question. Thus, this research plays a crucial role in providing valuable insights and concrete approaches for safeguarding Industrial Automation and Control Systems (IACSs) in an environment characterized by rapid technological evolution.

   \vspace{\onelineskip}
 
   \noindent 
   \textbf{Keywords}: Cybersecurity. OPC UA. Vulnerabilities. Attack. IACS.
 \end{otherlanguage*}
\end{resumo}
